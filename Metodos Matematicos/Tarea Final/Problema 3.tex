\documentclass{article}


\usepackage[sc]{mathpazo}
\usepackage{tcolorbox}
\usepackage[T1]{fontenc} 
\linespread{1.22} % Line spacing - Palatino needs more space between lines
\usepackage{microtype} % Slightly tweak font spacing for aesthetics
\usepackage[utf8]{inputenc}
\usepackage[spanish,es-tabla]{babel}

%\usepackage{multirow}
\usepackage[T1]{fontenc}
\usepackage{mathptmx}
%\usepackage{multicol}
\usepackage{amsmath}
\usepackage{float}
\usepackage{amssymb}
\usepackage{subfigure}
\usepackage{wrapfig}
\usepackage[siunitx]{circuitikz}
\usepackage{pdfpages}
\usepackage{verbatim}
\usepackage{listings}
\definecolor{miverde}{rgb}{0,0.6,0}
\definecolor{migris}{rgb}{0.5,0.5,0.5}
\definecolor{mimalva}{rgb}{0.58,0,0.82}
\definecolor{backcolour}{rgb}{0.95,0.95,0.92}
\lstset{  %
  backgroundcolor=\color{backcolour},
  basicstyle=\footnotesize,
  breakatwhitespace=false,
  breaklines=true,
  captionpos=t,
  commentstyle=\color{miverde},
  deletekeywords={...},
  escapeinside={\%*}{*)},
  extendedchars=true,
  frame=single,	                   
  keepspaces=true, 
  keywordstyle=\color{blue},
  language=Python,
  numbers=left,                    
  numbersep=5pt,
  numberstyle=\small\color{migris}, 
  rulecolor=\color{black},        
  showspaces=false, 
  showstringspaces=false, 
  showtabs=false, 
  stepnumber=2,
  stringstyle=\color{mimalva},
  tabsize=2,	
  title=\lstname  
}

\usepackage[hmarginratio=1:1,top=32mm,columnsep=20pt]{geometry} % Document margins
\usepackage[hang, small,labelfont=bf,up,textfont=it,up]{caption} 
\usepackage{booktabs}
\usepackage{lettrine}
\usepackage{enumitem} 
\setlist[itemize]{noitemsep} 
\usepackage{abstract}
\renewcommand{\abstractnamefont}{\normalfont\bfseries} 
\renewcommand{\abstracttextfont}{\normalfont\small\itshape}

\usepackage{titlesec} 
\renewcommand\thesection{\Roman{section}} 
\renewcommand\thesubsection{\roman{subsection}} 
\titleformat{\section}[block]{\large\scshape\centering}{\thesection.}{1em}{} 
\titleformat{\subsection}[block]{\large}{\thesubsection.}{1em}{} 

\usepackage{fancyhdr}
\pagestyle{fancy} 
\fancyhead{} 
\fancyfoot{} 
\fancyhead[L]{Problema 3} 
\fancyhead[R]{ Mayo 2023 } 
\fancyfoot[RO,LE]{\thepage}

\usepackage{lipsum}
\usepackage{titling} 
\usepackage{algorithm2e}
\usepackage{hyperref} \hypersetup{
    colorlinks=true,
    linkcolor=blue,
    filecolor=magenta,      
    urlcolor=cyan,
}

%----------------------------------------------------------------------------------------
%	TITLE SECTION
%----------------------------------------------------------------------------------------


\begin{document}

Ecuación de calor con frontera en estado no estacionario sobre un disco

\begin{equation}
\label{eq:1}
\frac{1}{r}\frac{\partial}{\partial r}\left(r\frac{\partial \Psi}{\partial r}\right)+\frac{1}{r^2}\frac{\partial^2 \Psi}{\partial \phi^2}=\frac{1}{\alpha^2}\frac{\partial \Psi}{\partial t}
\end{equation}
C.F.:
$$\Psi(r=0, \phi, t) = finito$$
$$\Psi(r=a, \phi, t) = f(\phi)$$
$$\Psi(r, \phi, t)=\Psi(r, \phi+2\pi, t)$$
C.I.:
$$\Psi(r, \phi, t=0)=g(r, \phi)$$

Supongamos que $\Psi$ es una función de la forma:
\begin{equation}
\label{eq:2}
\Psi(r, \phi, t)=\Psi_p(r, \phi, t) + \Psi_c(r, \phi)
\end{equation}

donde $\Psi_p$ es solución a la ecuación diferencial con condiciones de frontera homogéneas.

Sustituyendo a la ec. \ref{eq:2} en la ec. \ref{eq:1}, obtenemos:
\begin{equation*}
\frac{1}{r}\frac{\partial}{\partial r}\left(r\frac{\partial \Psi_p}{\partial r}\right)+\frac{1}{r^2}\frac{\partial^2 \Psi_p}{\partial \phi^2}+\frac{1}{r}\frac{\partial}{\partial r}\left(r\frac{\partial \Psi_c}{\partial r}\right)+\frac{1}{r^2}\frac{\partial^2 \Psi_c}{\partial \phi^2}=\frac{1}{\alpha^2}\frac{\partial \Psi}{\partial t}
\end{equation*}

Por lo tanto:
\begin{equation}
\frac{1}{r}\frac{\partial}{\partial r}\left(r\frac{\partial \Psi_c}{\partial r}\right)+\frac{1}{r^2}\frac{\partial^2 \Psi_c}{\partial \phi^2}=0
\end{equation}\label{eq:3}

Cuya solución general es:
\begin{equation}
\label{eq:4}
\Psi_c(r,\phi)=b_0 ln(r)+a_0+\sum_{n=1}^{\infty} (a_n r^n \cos(n\phi)+b_nr^n\sin(n\phi))+\sum_{n=1}^{\infty} (c_n r^{-n} \cos(n\phi)+d_nr^{-n}\sin(n\phi))
\end{equation}

Dadas las condiciones de frontera:
$$\Psi(r=0, \phi, t)=\Psi_p(r=0, \phi, t)+\Psi_c(r=0, \phi)=finito$$
dado que:
$$\Psi_p(r=0, \phi, t)=finito$$
$$\Rightarrow \Psi_c(r=0, \phi)=finito$$

Esto hace que $\Psi_c$, sea
\begin{equation*}
\Psi_c(r,\phi)=a_0+\sum_{n=1}^{\infty} (a_n r^n \cos(n\phi)+b_nr^n\sin(n\phi))
\end{equation*}

Por la otra condición de frontera
$$\Psi(r=a, \phi, t)=\Psi_p(r=a, \phi, t)+\Psi_c(r=a, \phi)=f(\phi)$$
$$0+\Psi_c(r=a, \phi)=f(\phi)$$
$$\Rightarrow a_0+\sum_{n=1}^{\infty} (a_n a^n \cos(n\phi)+b_na^n\sin(n\phi))=f(\phi)$$
$$\Rightarrow A_0+\sum_{n=1}^{\infty} (A_n \cos(n\phi)+B_n\sin(n\phi))=f(\phi)$$
$$ A_0 = a_0, \quad A_n = a_na^n, \quad B_n = b_n a^n$$
$$ \quad a_n = \frac{A_n}{a^n}, \quad b_n = \frac{B_n}{a^n}$$

$$\Rightarrow A_0+\sum_{n=1}^{\infty} \left(\frac{r}{a}\right)^n(A_n \cos(n\phi)+B_n\sin(n\phi))=f(\phi)$$
de esto, al ser una serie de Fourier, los coeficientes son:
$$A_0=\frac{1}{2\pi}\int_0^{2\pi}f(\phi)d\phi$$
$$A_n=\frac{1}{\pi}\int_0^{2\pi} f(\phi)\cos(n\phi)d\phi$$
$$B_n=\frac{1}{\pi}\int_0^{2\pi} f(\phi)\sin(n\phi)d\phi$$

Teniendo esto, llegamos a que la solución completa y general para $\Psi$ es:
\begin{tcolorbox}
\begin{equation}
\begin{split}
\Psi(r,\phi,t)=&\sum_{n=1}^{\infty} \sum_{m=1}^{\infty} J_n(\lambda_{nm}r)[D_{nm}\cos(n\phi) + E_{nm} \sin(n\phi)]e^{-\lambda_{nm}^2 \alpha^2 t}\\
&+\sum_{n=1}^{\infty} \left(\frac{r}{a}\right)^n(A_n \cos(n\phi)+B_n\sin(n\phi))+A_0
\end{split}
\end{equation}
\end{tcolorbox}

Por último, aplicando la C.I., $\Psi(r, \phi, t=0)=g(r, \phi)$
\begin{equation}
\begin{split}
\Psi(r,\phi,t=0)=&\sum_{n=1}^{\infty} \sum_{m=1}^{\infty} D_{nm}\cos(n\phi)J_n(\lambda_{nm}r)\\
&+\sum_{n=1}^{\infty} \sum_{m=1}^{\infty} E_{nm}\sin(n\phi)J_n(\lambda_{nm}r)\\
&+\sum_{n=1}^{\infty} \left(\frac{r}{a}\right)^n (A_n \cos(n\phi)+B_n\sin(n\phi))+A_0=g(r,\phi)
\end{split}
\end{equation}
$$\Rightarrow \sum_{n=1}^{\infty} \sum_{m=1}^{\infty} D_{nm}\cos(n\phi)J_n(\lambda_{nm}r)
+\sum_{n=1}^{\infty} \sum_{m=1}^{\infty} E_{nm}\sin(n\phi)J_n(\lambda_{nm}r)=g(r,\phi)-\Psi_c(r,\phi)$$

$$\Rightarrow \sum_{n=1}^{\infty} \cos(n\phi)\sum_{m=1}^{\infty} D_{nm} J_n(\lambda_{nm}r)
+\sum_{n=1}^{\infty} \sin(n\phi) \sum_{m=1}^{\infty} E_{nm} J_n(\lambda_{nm}r)=g(r,\phi)-\Psi_c(r,\phi)$$

$$\Rightarrow \sum_{n=1}^{\infty} \cos(n\phi)G_1(r)
+\sum_{n=1}^{\infty} \sin(n\phi)G_2(r)=g(r,\phi)-\Psi_c(r,\phi)$$

Donde:
$$G_1(r)=\sum_{m=1}^{\infty} D_{nm} J_n(\lambda_{nm}r)$$ 
$$G_2(r)=\sum_{m=1}^{\infty} E_{nm} J_n(\lambda_{nm}r)$$
\begin{equation*}
\begin{split}
&G_1=\frac{1}{\pi}\int_0^{2\pi} cos(n\phi)[g-\Psi_c]d\phi\\
&G_2=\frac{1}{\pi}\int_0^{2\pi} sin(n\phi)[g-\Psi_c]d\phi
\end{split}
\end{equation*}
Por ende,
\begin{equation}
\begin{split}
D_{nm}&=\frac{\left< \frac{1}{\pi} \int_{0}^{2\pi} cos(n\phi)[g-\Psi_c]d\phi | J_n(\lambda_{nm}r) \right> }{\langle J_n(\lambda_{nm}r) | J_n(\lambda_{nm}r) \rangle} = \frac{\frac{1}{\pi} \int_0^a \int_0^{2\pi} cos(n\phi)[g-\Psi_c] J_n(\lambda_{nm}r) drd\phi }{\int_0^a [J_n(\lambda_{nm}r)]^2 rdr }\\
&=\frac{\frac{1}{\pi} \int_0^a \int_0^{2\pi} cos(n\phi)[g-\Psi_c] J_n(\lambda_{nm}r) drd\phi }{ \frac{1}{2}a^2 J_{n+1}^2 (\lambda_{nm}a) }
\end{split}
\end{equation}
\begin{tcolorbox}
$$\therefore D_{nm}=\frac{2}{\pi a^2 J_{n+1}^2 (\lambda_{nm}a)} \int_0^a \int_0^{2\pi} cos(n\phi)[g-\Psi_c] J_n(\lambda_{nm}r) drd\phi $$
\end{tcolorbox}

\begin{equation}
\begin{split}
E_{nm}&=\frac{\left< \frac{1}{\pi} \int_{0}^{2\pi} sin(n\phi)[g-\Psi_c]d\phi | J_n(\lambda_{nm}r) \right> }{\langle J_n(\lambda_{nm}r) | J_n(\lambda_{nm}r) \rangle} = \frac{\frac{1}{\pi} \int_0^a \int_0^{2\pi} sin(n\phi)[g-\Psi_c] J_n(\lambda_{nm}r) drd\phi }{\int_0^a [J_n(\lambda_{nm}r)]^2 rdr }\\
&=\frac{\frac{1}{\pi} \int_0^a \int_0^{2\pi} sin(n\phi)[g-\Psi_c] J_n(\lambda_{nm}r) drd\phi }{ \frac{1}{2}a^2 J_{n+1}^2 (\lambda_{nm}a) }
\end{split}
\end{equation}
\begin{tcolorbox}
$$\therefore E_{nm}=\frac{2}{\pi a^2 J_{n+1}^2 (\lambda_{nm}a)} \int_0^a \int_0^{2\pi} sin(n\phi)[g-\Psi_c] J_n(\lambda_{nm}r) drd\phi $$
\end{tcolorbox}
%----------------------------------------------------------------------------------------

\end{document}