\documentclass[12pt]{article}
	
%______________________PREAMBULO_________________________

%----------------------Paquetes--------------------------
\usepackage{amsmath,amssymb,amsfonts,latexsym,cancel} % Paquetes de símbolos adicionales.
\usepackage[spanish,es-tabla]{babel} % Idioma español
\usepackage[utf8]{inputenc} % Paquete que nos permite usar los acentos y otros símbolos, directamente del teclado.
\usepackage[T1]{fontenc} % Cambia el tipo de letra
\usepackage{times} % Tipo de letra Times New Roman
\usepackage{graphicx} % Paquete para el manejo de gráficos y figuras en el documento.
\usepackage{geometry} % Permite el manejo de los margenes
\usepackage{fancyhdr} % Permite colocar y manejar el encabezado
\usepackage[breaklinks,colorlinks=true,linkcolor=black,citecolor=blue, urlcolor=blue]{hyperref} % Crea hipervinculo entre secciones y el indice
\usepackage{pstricks}
%\usepackage{multicol}
%\usepackage{mathpazo} %fuente palatino
%\usepackage{xcolor}
%\usepackage[shortlabels]{enumitem}
%-------------Paquetes para el formato de las citas-------
%\usepackage[hyphens]{url}
%\usepackage{float}
%\usepackage{cite}
%\usepackage{wrapfig}

%-----------------------------ayuda de paquetes--------------------

\spanishdecimal{.}

%------------------------Margenes----------------------------

\newgeometry{bottom = 2.5 cm, top = 2.5 cm, left = 2 cm, right = 2 cm} % Modifica el margen {Abajo, Arriba, Izquierda, Derecha

%----------------------------Interlineado----------------------------------

%\doublespacing
%\onehalfspace
%\singlespace
%\spacing{1.5} % Permite personalisar a gusto
%\setlength{\parskip}{2cm} % Es el espacio entre parrafos

%-----------------------------Sangria---------------------------------------

\setlength{\parindent}{0 cm} % Manipula la sangria

%---------------------Portada------------------

%\title{
%\begin{figure}[h!]
		
%	\centering
%	\includegraphics[width=\linewidth]{Nom_UAdeC_FCFM.png}  			
			
%\end{figure}
%\huge \textbf{LABORATORIO DE FISICA 3}\\\LARGE TITULO PRACTICA\\}
%\author{ \Large \textbf{Profesor:}\\
%\Large \textbf{Alumno:} Oscar Joel Castro Contreras}
%\date{\today}

%--------------Encabezado y pie de pagina--------------------

\pagestyle{fancy}%Coloca el encabezado en el documento
\lhead[]{Física 3}%Encabezado izquierda
\rhead[]{Oscar Joel Castro Contreras}%Encabesado derecha
%\chead[]{}%Encabesado central
\renewcommand{\headrulewidth}{0.08 pt}%Coloca linea al pie de pagina

%\lfoot[]{PI}%Pie de pagina izquerdo
%\rfoot[]{PD}%Pie de pagina derecho
\cfoot[]{\thepage}%Pie de pagina central
\renewcommand{\footrulewidth}{0.08 pt}%Coloca linea al pie de pagina

%-----------------------------------------------------------------------------

	\begin{document}
		
		\begin{titlepage}
		
			\centering
			{\bfseries
			\begin{figure}[h!]
				\centering
				\includegraphics[width=\linewidth]{Nom_UAdeC_FCFM.png} 				
			\end{figure}
			\par}
			\vspace{2cm}
			{\scshape\LARGE FÍSICA 3 \par}
			\vspace{3cm}
			{\scshape\Huge \textbf{Tarea 2 - Ley de Gauss} \par}
			\vfill
			{\LARGE \textbf{Profesora:} Ricardo Pérez Martinez \par}
			\vspace{3cm}
			{\LARGE \textbf{Alumno:} Oscar Joel Castro Contreras \par}
			\vfill
			{\Large \today \par}
			\thispagestyle{empty}
			%\thispagestyle{fancy}
			
		\end{titlepage}
	
		\newpage
		
		\tableofcontents		
		
		\newpage
		
		\section*{Formulas:}\label{sec:Formulas}
			$$ k_e = 8.99 \times 10^9 \frac{Nm^2}{C^2} \quad \epsilon_0 = 8.8542 \times 10^{-12} \frac{C^2}{Nm^2} $$
			$$ \rho = \frac{Q}{V}, \quad \sigma = \frac{Q}{A}, \quad \lambda = \frac{Q}{L} $$
			$$ dq = \rho dV, \quad dq = \sigma dA, \quad dq = \lambda dL $$
			\begin{align}

			\end{align}

		\section{Problema 1:}\label{sec:Problema1}
			Sobre la superficie de una coraza esférica aislante de radio $ R $, está distribuida con
			uniformidad una carga negativa $ -Q $. Calcula la fuerza (magnitud y dirección) que
			ejerce la coraza sobre una carga puntual positiva $ q $ ubicada a una distancia:
			\begin{enumerate}
				\item[a)] $ r > R $ del centro de la coraza (fuera de la coraza).
				\item[b)] $ r < R $ del centro de la coraza (dentro de la coraza).
			\end{enumerate}

		\section{Problema 2:}\label{sec:Problema2}
			Un conductor cílindrico de longitud infinita tiene un radio $ R $ y densidad superficial
			de carga uniforme. a) En terminos de  y R. >cual es la carga por unidad de
			longitud  para el cilindro?, b) en terminos de , >cual es la magnitud del campo
			electrico producido por el cilindro con carga a una distancia r > R de su eje?,
			c) expresa el resultado del enciso b) en terminos de  y demuestra que el campo
			electrico fuera del cilindro es el mismo que si toda la carga estuviera sobre el eje.
			Compara tu resultado con el que se obtuvo en clase para una lnea innita de carga.


		\section{Problema 3:}\label{sec:Problema3}

		\section{Problema 4:}\label{sec:Problema4}

		\section{Problema 5:}\label{sec:Problema5}

		\section{Problema 6:}\label{sec:Problema6}
			
		\section{Problema 7:}\label{sec:Problema7}

		\section{Problema 8:}\label{sec:Problema8}

		\section{Problema 9:}\label{sec:Problema9}

		\section{Problema 10:}\label{sec:Problema10}

	\end{document}