\documentclass[12pt, letterpaper]{article} %

%_____________________________PREAMBULO_______________________________

%-----------------------------Paquetes--------------------------------

\usepackage{amsmath,amssymb,amsfonts,latexsym} %Esta instrucción indica que en este documento se usarán paquetes de símbolos adicionales.
\usepackage[spanish,es-tabla]{babel} % Idioma español
\usepackage[utf8]{inputenc} % Paquete que nos permite usar los acentos y otros símbolos, directamente del teclado.
%\usepackage[T1]{fontenc} %Cambia el tipo de letra
\usepackage[utf8]{inputenc}%Esta instrucción se usa para incluir un paquete que nos permite usar los acentos y otros símbolos, directamente del teclado.
\usepackage{graphicx}%Esta instrucción se usa para incluir un paquete para el manejode gráficos y figuras en el documento.
\usepackage{geometry}%Permite el manejo de los margenes
\usepackage{fancyhdr}%Permite colocar y manejar el encabezado
\usepackage[breaklinks,colorlinks=true,linkcolor=black,citecolor=blue, urlcolor=blue]{hyperref} % Crea un hipervinculo de las secciones con el indice
\usepackage{setspace} % Ayuda con el interlineado del texto, tambien se puede usar por parrafos
\usepackage{multicol} % Ayuda a dividir el texto en distintas columnas

%-----------------------------ayuda de paquetes--------------------

\spanishdecimal{.} % Coloca puntos enves de comoas en un espacio matematico

%-----------------------------Margenes-------------------------------------

%Margen base \newgeometry{bottom = 0 cm, top = -1.5 cm, left = 0 cm, right = 0 cm}
\newgeometry{bottom = 2.5 cm, top = 2.5 cm, left = 3 cm, right = 3 cm} %Modifica el margen {Abajo, Arriba, Izquierda, Derecha

%----------------------------Interlineado----------------------------------

%\doublespacing
%\onehalfspace
%\singlespace
%\spacing{1.5} % Permite personalisar a gusto
%\setlength{\parskip}{2cm} % Es el espacio entre parrafos

%-----------------------------Sangria---------------------------------------

\setlength{\parindent}{0 cm} % Manipula la sangria

%----------------------------Portada----------------------------------

%\title{MI PRIMER DOCUMENTO}% Aqui va el titulo del documento
%\author{Oscar Joel Castro Contrera \thanks{LaTex}}% Aqui va el nombre del autor y los agradecimientos
%\date{\today}% Esto corresponde a la fecha del dia

%---------------------Encabezado y pie de pagina-------------------------------------

\pagestyle{fancy}%Coloca el encabezado en el documento
\lhead[]{\includegraphics[width=1cm]{demon.png} }%Encabezado izquierda
\rhead[]{Nombre}%Encabesado derecha
\chead[]{}%Encabesado central
\renewcommand{\headrulewidth}{0.08 pt}%Coloca linea al pie de pagina

\lfoot[]{PI}%Pie de pagina izquerdo
\rfoot[]{PD}%Pie de pagina derecho
\cfoot[]{\thepage}%Pie de pagina central y \thepage crea la enumeracion de las paguinas
%\renewcommand{\footrulewidth}{0.08 pt}%Coloca linea al pie de pagina

%-----------------------------------------------------------------------------

\begin{document}% Con esta instruccion se inicia un documento
	
	%\maketitle % Introduce todo lo que esta fuera del \begin{document} dentro del documento

	\begin{titlepage}
		\centering
		{\bfseries
		\begin{figure}[h!]
			\centering
			\includegraphics[width=\linewidth]{Nom_UAdeC_FCFM.png}  						
		\end{figure}
		\par}
		\vspace{2cm}
		{\scshape\LARGE LABORATORIO DE FISICA 3 \par}
		\vspace{3cm}
		{\scshape\Huge \textbf{TÍTULO PRACTICA} \par}
		\vfill
		{\LARGE \textbf{Profesor:} \par}
		\vspace{3cm}
		{\LARGE \textbf{Alumno:} Oscar Joel Castro Contreras \par}
		\vfill
		{\Large \today \par}
		\thispagestyle{empty} % Quita los Encabezados y pies de pagina y enumerado de la portada
		%\thispagestyle{fancy} % Incluye el encabesado y el pie de paguina en la portada
	\end{titlepage}
     
     \newpage % Crea una nueva pagina
     
     \tableofcontents % Crea un indice con las secciones y subsecciones creadas
	
	\newpage
	
     \begin{abstract}% Con esta instriccion se agrega el resumen del articulo
     
          Este parrafo se escribe el resumen de nuestro articulo cientifico.
          
     \end{abstract}
	
          En este documento hay algunos paquetes y parametros adicionales. Hay paquete de codificacion con parametros de tamaño de pagina y tamaño de fuente.
		
          Esta linea comenzara un segundo parrafo, y se puede romper una \\ las lineas \\ y continuar con 2 lineas\\\\
		
	\section{Negritas, Cursiva, Subrayado, etc.}\label{sec:Negritas, Cursiva, Subrayado, etc.} Crea una referencia a una seccion
		
          \textbf{Con esta instruccion el texto tomo el formato de negritas.}\\
          \underline{Con esta instruccion el texto tomo el formato de subrayado.}\\
          \textit{Con esta instruccion el texto toma el formato de cursiva.}\\
          \textbf{\textit{En esta parte se unieron las 2 intrucciones de las anteriores.}}\\
          \emph{Con esta instruccion el texto tambien toma el formato de cursiva.}\\\\

          EJEMPLOS:\\\\
          
	\begin{enumerate} % este entormo enumeroa una lista
	
          \item Algunos de los \textbf{mayores} descubrimientos en la ciencia fueron por accidente.\\\\
          \item Algunos de los mejores descubrimentos en la \underline{ciencia} fueron por accidente.\\\\
          \item Algunos de los mayores \emph{descubrimientos} en la ciencia fueron por accidente.\\\\
          \item \textit{Algunos de los mayores \emph{descubrimientos} en la ciencia fueron por accidente.}\\\\
          \item \textbf{Algunos de los mayores \emph{descubrimientos} en laciencia fueron por accidente.}\\\\          
          \item \textit{Algunos de los mayores \emph{descubrimientos} en laciencia fueron por accidente.}\\\\
          
	\end{enumerate}
	
	\section{Matematicas}\label{sec:Matematicas}
		
		\subsection{Ecuacion Cuadratica}
	
			$$y = x^2+5x+6$$
		
		\subsection{Integral}
		
			$$\int f(x) dx$$
			
		\subsection{Ecuaciones de flujo de carga electrica}
			
			$$ P_i =  V_i \sum_{j=1}^{n} V_j \left( G_{ij} \cos\delta_{ij}+B_{ij} \sen\delta_{ij} \right) $$
			
			Seguna la ecuacion PA \ref{eq:PotenciaActiva}, podemos ...............			
			
			\begin{equation}\label{eq:PotenciaActiva}
				P_i =  V_i \sum_{j=1}^{n} V_j \left( G_{ij} \cos\delta_{ij}+B_{ij} \sen\delta_{ij} \right)
			\end{equation}

			\begin{equation}\label{eq:PotenciaActiv}
				P_i =  V_i \sum_{j=1}^{n} V_j \left( G_{ij} \cos\delta_{ij}+B_{ij} \sen\delta_{ij} \right)
			\end{equation}
			
		\subsection{Flechas}
			
			$ \rightarrow $
			$ \longrightarrow $
			$ \leftarrow $
			
		
	\section{Imagenes}\label{sec:Imagenes}
	
		Comprobando seccion\ref{sec:Negritas, Cursiva, Subrayado, etc.}   \textit{figura} \ref{fig:LOGO}, referencia de imagen \ref{fig:OSCAR} % crea una referenca y un 				hipervinculo a la figura que se coloca dentro de los corchetes
	
		% [t] = top: Pone imagen en parte superior
		% [b] b = bottom: Pone imagen en parte inferir
		% [h] h = here: pone donde esta escrita
		% [+!] +!: coloca exacto donde se escribe la imagen
		\begin{figure}[!]
			\centering % Centra la imagen
			\includegraphics[width=\linewidth]{Nom_UAdeC_FCFM.png} % Coloca una imagen [width=\linewidth] coloca la imagen dentro de los margenes [scale=1] es el tamaño
			\caption{Logo de UA de C y FCFM} % Agrega un pie de figura
			\label{fig:LOGO} % Se le coloca un nombre para poder referenciarla
		
		\end{figure}
		
		Que loco no cren

		\begin{figure}
			\centering % Centra la imagen
			\includegraphics[width=.5\linewidth]{OSCAR F.jpg} 
			\caption{Fondo OSCAR} % Agrega un pie de figura
			\label{fig:OSCAR} % 
		
		\end{figure}
		
	\section{Tablas}\label{sec:Tablas}

	\begin{table}[h!]% [] igual que las imagenes
		\centering
		\begin{tabular}{|cc|c||c|} %{l} alineacion izquierda, {r} alineacion derecha, {c} alineacion centro
		
			\hline
			Celda1 & Celda2 & Celda3 & Celda4 \\\hline
			Celda5 & Celda6 & Celda7 & Celda8 \\\hline
			\hline
			\hline
			
		\end{tabular}
		\caption{Tabla new}
		\label{tab:1}

	\end{table}
	
	Hacemos referencia a la tabla \ref{tab:1} que contiene 2 filas y 4 columnas
	
	segun el libro \cite{bib:item1} pasa algo......................
	\section{Bibliografia}\label{sec:Bibliografia}
	
		\begin{thebibliography}{20}
			
			\bibitem{bib:item1} \textsc{Autores} \textit{Nombre libro}
			\bibitem{bib:item2} \url{https://www.youtube.com/watch?v=zqXU1LOpT_8}

		\end{thebibliography}
	 
\end{document}% Con esta instruccion se finaliza un documento