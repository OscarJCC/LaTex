<\documentclass[11pt,letter]{article} %article,report,book


     % --- inicia preambulo ----------

          % --- Paquetes -------------

          \usepackage[utf8]{inputenc} % Opciones encodig: utf8, latin1
          %\texttt{\usepackage{amsmath,amssymb,amsfonts,latexsym}}
          %\usepackage{graphicx}
          \usepackage{float}
          \usepackage[spanish]{babel}
          \usepackage{eurosym}


         % \usepackage[total={18cm,21cm},top=2cm,left=2cm]{geometry}
          %\usepackage[a4paper,total={6.5in,8.75in},top=1.2in,left=0.9in,includefoot]{geometry}

          % --- Fin de paquetes ------


     \title{Segundo docuemtno de LaTex}
     \author{Oscar Joel Castro Contreras}
     \date{\today}

     % --- fin preambulo -------------

     \begin{document}

          \maketitle %Coloca lo que esta en el preambulo

          %\newpage %Crea un pagina

          %\tableofcontents %Crea un indice

          %\newpage

          \begin{abstract}
          A continuación se presenta un documento con características 
          de edición en \LaTeX, éste servirá como ejemplo para introducir 
          algunos comando básicos, sin embargo estos ejemplos y parte 
          del contenido fueron obtenidos del documento llamado 
          \textbf{Herramientas Informáticas de las Matemáticas e Ingeniería} 
          cuyo autor es \textit{Ernesto Aranda Ortega}.
          \end{abstract}

          \section{Aspectos Generales}
          \label{sec:aspgenerales} %etiqueta para hacer referencia a la seccion

          Los aspectos generales son la base para comprender la filosofía 
          de edición en \LaTeX, tal vez esto es principalmente uno de 
          los primeros pasos complicados del lenguaje. Sin embargo con 
          la práctica continua, es casi seguro que sin darse cuenta 
          el programador (editor) en un cierto tiempo escribirá de manera 
          automática muchos de estos aspecto.

          \subsection{Caracteres especiales} 
          \label{sec:caracteresesp}

          Los siguientes caracteres tiene un significado especial para 
          el compilador \TeX:

          \begin{table}
               \centering
               \begin{tabular}{cl}
                    \textbackslash   & Carácter inicial de los comandos \TeX, por ejemplo, \verb|\alpha| , \verb|\section| , \verb|\bf| , etc. \\
                         \verb|\$|  & \$ Delimitador del modo matemático. \\
                         \verb|\%|  & \% \\
                         \verb|\^{}|  & \^{} Carácter de pueríndice en modo matemático.\\
                         \verb| \_|  & \_ Carácter de subíndice en el modo matemático. \\
                         \verb|\{|  & \{ Apertura de delimitador. \\
                         \verb| \}|  & \}  Cierre de delimitador. \\
                         \verb|\~|  & \~{} \\
                         \verb|\#|  & \# \\
               \end{tabular}
               \caption{Caracteres especiales que se deben tener encuenta. Fuente: Elaboración propia.}
          \end{table}

          \subsection{Ordenes en \LaTeX}
          \label{sec:ordenes}

          Las órdenes en \LaTeX{} son sensibles a mayúsculas y comienzan 
          con una retrobarra \ y tienen un nombre con puras letras. 
          Si quieres un espacio en blanco después de una orden, se tiene 
          que poner \{ \} despues de la orden o una orden especial de espaciado.

          \subsection{Acentos}
          \label{sec:acentos}

          El fichero fuente de \LaTeX{} sólo reconoce caracteres del alfabeto 
          inglés. Para escribir acentos tenemos dos opciones:

          \begin{itemize}
          \item Uso de comandos que producen acentos y letras no inglesas.
               \begin{table}[H]
                    \centering
                    \begin{tabular}{c|c}
                         \hline
                         \verb|\'a| $\rightarrow $ á   & \verb|\'e| $\rightarrow $ é \\
                         \verb|\'\i| $\rightarrow $ í   & \verb|\'e| $\rightarrow $ é \\
                         \verb|\'A| $\rightarrow $ Á   & \verb|\'E| $\rightarrow $ É \\
                         \verb|\'o| $\rightarrow $ ó   & \verb|\"o| $\rightarrow $ \"o \\
                         \verb|\c{c}| $\rightarrow $ \c{c}   & \verb|\c{C}| $\rightarrow $ \c{C} \\
                         \verb|\`e| $\rightarrow $ \`e   & \verb|\v o| $\rightarrow $ \v o \\
                         \verb|\~n| $\rightarrow $ \~n   & \verb|\~N| $\rightarrow $ \~N \\
                         \verb|?`| $\rightarrow $ ?`   & \verb|!`| !` $\rightarrow $ é \\
                         \hline
                    \end{tabular}
                    \caption{Acentos en \LaTeX. Fuente: Elaboración propia}
                    \label{tab:acentos}
               \end{table}
          \item Uso del paquete \verb|\usepackage [latin1]{inputenc}|
          \end{itemize}

          \subsection{Otro símbolos}
          \label{sec:otrossimbolos}

          El uso de comillas dobles, simples, guiones, ordinales, puntos 
          suspensivos y otros símbolos está ilustrado en el siguiente texto.

          Las comillas < < dobles > > o < < francesas > > difieren de las 
          ''inglesas'', o las comillas 'simples'. Los guiones pueden ser -cortos-, 
          --medios-- o ---largos---, y los puntos suspensivos ... a veces son más 
          cortos y a veces más largos \dots Hay infinidad de símbolos adicionales 
          que no es necesario aprender, como \copyright, los ordinales 1\textsuperscript{a}, 
          3\textsuperscript{er}, 34\textsuperscript{o} o el símbolo del euro \euro, 
          para el que es necesario el uso del paquete {\bf eurosym}

          El parrafo anterior se ha generado con:

          \begin{verbatim}
          Las comillas < < dobles > > o < < francesas > > difieren de las ''inglesas'',
          o las comillas 'simples'. Los guiones pueden ser -cortos-, --medios-- o ---largos---,
          y los puntos suspensivos ... a veces son más cortos y a veces más largos \dots
          Hay infinidad de símbolos adicionales que no es necesario aprender, como \copyright,
          los ordinales 1\textsuperscript{a}, 3\textsuperscript{er}, 34\textsuperscript{o} o 
          el símbolo del euro \euro, para el que es necesario el uso del paquete {\bf eurosym}

          \end{verbatim}

     \end{document}
